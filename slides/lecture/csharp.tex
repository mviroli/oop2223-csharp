\documentclass[presentation]{beamer}
\usepackage{common}

\title[\lecturecode{C\#}]{Object-oriented programming in C\#}

\begin{document}

\frame[label=coverpage]{\titlepage}

\fr{Outline}{
  \bl{Parts}{\en{
    \item Basic OO concepts in C\#
    \item Some specific C\#/.NET mechanisms
    \item Encapsulation, interfaces, inheritance
    \item Generics
    \item Exceptions and some key C\# libraries
    \item Functional programming in C\#
    \item Other programming mechanisms
  }}
}

\section{Basic OO in C\#}
\newcommand{\codepath}[1]{../../code/lecture-01/#1}

\frs{5}{Brief introduction to C\#}{
  \bl{C\# and .NET}{\iz{
    \item Designed by Anders Hejlsberg around 2000 at Microsoft
    \item Is part of the .NET initiative, designed to compile over the Common Language Infrastructure (CLI)
    \item Current version is 9.0, released in 2020 for .NET 5.0
    \item Mono is a free, open-source compiler and runtime environment
    \item Initially developed as very similar to Java, then somewhat diverged
    \item Essentially, C\# took a different path than Java in following Scala
    \item Shall in these slides refer to ``mainstream/standard OOP'' to mean the intersection of Java/C\#
    }}
}

\frs{5}{.NET}{
  \bl{Main elements}{\iz{
    \item .NET started as a ployglot framework since its beginning
    \item C\# is by far the mostly used language
    \item Concepts replicate Java and JVM: CIL/bytecode, CLR/JVM, and so on
    \item As a key difference, .NET initially targets Microsoft Windows
    }}
  \fg{height = 0.4\textheight}{img/dotnet-overview.pdf}
}


\begin{frame}{\dotnet Platform -- Present vs. Past}
 \bl{Past to Present}{\iz{
        \item Before \dotnet 5 there used to be three major implementations of the \emph{class library}:
        %
        \begin{description}
            \item[\dotnet Framework] | Windows-specific, full-featured, targetting desktop and web applications
            \item[\dotnet Core] | multi-platform (Win, Mac, Linux), less-featured, targetting desktop and web applications
            \item[Xamarin] | mobile-oriented (Android, iOS, Mac OS) 
        \end{description}

        \vfill

        \item Since \dotnet 5, implementations are aligned
    }}
    \bl{In these slides}{
        Stick to \alert{\dotnet Core 3.1}, to maximise interoperability and to avoid compatibility issues
    }
\end{frame}


\frs{20}{Features of C\#}{
  \bl{Ingredients}{\iz{
    \item C-like language: the imperative and structured part are very similar
    \item Java-like language: essentially very similar to Java, specially at the beginning
    \item Static and strong typing: types are checked at run-time, preventing ill-typed operations
    \item Object-orientation: object by references, automatic garbage collection
    \item Functional-orientation: extension methods, generics, delegates, lambdas
    }}
  \bl{Philosophy}{\iz{
    \item aiming at high expressiveness and richness, though become a rather ``large'' language
  }}
  \fg{height = 0.4\textheight}{img/sizes.png}
}


\fr{C\# types: we start with Simple Types and Class Types}{
  \fg{height = 0.6\textheight}{img/type-system.pdf}
}
  
\frs{20}{Variables: initialisation and inference}{
  \bl{On variables -- essentially as in Java}{\iz{
  \item Same rules on scoping, and assignment
  \item Similar distinction between primitive and class types
  \item Similar naming conventions for variables
  \item \texttt{null} is assignable to variables of reference types
  \item Can use \texttt{var} to declare a variable with type to be inferred
  \item Keywords (\cil{int}, \cil{bool}, \cil{string}, \cil{object}) map to Library Value Types or Classes
  }}
  \codeview{3}{9}{24}{\scriptsize}{\codepath{InitialExamples/Initialisation.cs}}
}

\begin{frame}{\dotnet Built-in Types}\centering
    \begin{table}[]
        \resizebox{\textwidth}{!}{
        \begin{tabular}{c|c|c|c|c}
            \textbf{Name}    & \textbf{Keyword} & \textbf{Category} & \textbf{Size} & \textbf{Description}  \\
            \hline\hline
            \texttt{Boolean} & \texttt{bool}    & \emph{val} & 1             & either \texttt{true} or \texttt{false} \\
            \texttt{Char}    & \texttt{char}    & \emph{val} & 2             & UTF-16 characters \texttt{`U+0000'} \ldots \texttt{`U+FFFF'} \\
            \texttt{Byte}    & \texttt{byte}    & \emph{val} & 1             & integers in $0 \ldots (2^{8}-1)$ \\
            \texttt{SByte}   & \texttt{sbyte}   & \emph{val} & 1             & integers in $-2^{7} \ldots (2^{7}-1)$ \\
            \texttt{Int16}   & \texttt{short}   & \emph{val} & 2             & integers in $-2^{15} \ldots (2^{15}-1)$ \\
            \texttt{UInt16}  & \texttt{ushort}  & \emph{val} & 2             & integers in $0 \ldots (2^{16}-1)$ \\
            \texttt{Int32}   & \texttt{int}     & \emph{val} & 4             & integers in $-2^{31} \ldots (2^{31}-1)$ \\
            \texttt{UInt32}  & \texttt{uint}    & \emph{val} & 4             & integers in $0 \ldots (2^{32}-1)$ \\
            \texttt{Int64}   & \texttt{long}    & \emph{val} & 8             & integers in $-2^{63} \ldots (2^{63}-1)$ \\
            \texttt{UInt64}  & \texttt{ulong}   & \emph{val} & 8             & integers in $0 \ldots (2^{64}-1)$ \\
            \texttt{Float}   & \texttt{float}   & \emph{val} & 4             & abs in $1.5\times 10^{-45} \ldots 3.4\times 10^{38}$ \\
            \texttt{Double}  & \texttt{double}  & \emph{val} & 8             & abs in $5.0\times 10^{-324} \ldots 1.7\times 10^{308}$ \\
            \texttt{Decimal} & \texttt{decimal} & \emph{val} & 16            & abs in $1.0\times 10^{-28} \ldots 7.9228 \times 10^{28}$ \\
            \texttt{Object}  & \texttt{object}  & \emph{ref} & O(1)          & anything \\
            \texttt{String}  & \texttt{string}  & \emph{ref} & O($n$)   & sequences of $n$ UTF-16 characters \\
        \end{tabular}
        }
    \end{table}

    \tiny
    (cf. \url{https://docs.microsoft.com/dotnet/csharp/language-reference/builtin-types/built-in-types})
\end{frame}


\frs{5}{C\# classes}{
  \bl{The core of OOP is essentially as in Java} {\iz{
    \item Classes, methods, fields, and constructors have same syntax and semantics
    \item Class instantiation, method invocation, field access have same syntax and semantics
    \item Static, non-static fields/methods have same syntax/semantics
    \item A source file must define the namespace (similar to Java package)
    \item Syntax for calling a constructor from another constructor is different
  }}
  \bl{Formatting}{\iz{
    \item Slightly different conventions on formatting braces 
    \item Methods start with an uppercase, fields with an underscore
    \item \myurl {https://docs.microsoft.com/en-us/dotnet/csharp/programming-guide/inside-a-program/coding-conventions}
  }}
}

\frs{5}{\Cil{Point3D}}{
\codeview{0}{2}{34}{\tiny}{\codepath{Point3D/Program.cs}}
}

\fr {C\# executable programs} {
   \bl {Building blocks of C\# software} {\iz {
     \item class libraries shipped with .NET
     \item possibly other external libraries 
     \item a set of classes that make up the application we build (like \cil{Point3D)}
     \item at least one of these classes has a special method \cil{Main}
     \item a \cil {Main} is the entry point of a program
   }}
   \bl{The \cil {Main} must have the following declaration: }{\iz {
   \item \cil {public static void Main() \{.. \}}
    \item there could be variant with different inputs, outputs, and visibility, but we won't see them now
    \item it is key it is call \cil{Main} and is \cil{static}
    \item \cil{static} means this method is ``shared'' among all objects, and is conceptually called to the class, not to the object
   }}
}

\frs{15}{Typical structure of an executable project} {
   \bl{Entry point class} {\iz {
     \item it contains the \cil{Main} method
     \item typically, it contains only that method
   }}
   \bl{Other classes}{\iz{
     \item contain the various application classes
   }}
   \bl{Source files}{\iz{
    \item have \cil{.cs} extension
    \item start with ``\cil{using}'' clauses to declare other classes they use
    \item declared one or more classes, enclosing them in \cil{namespace}s
    \item a \cil{namespace} is a ``module'' giving a context to the class
   }}
   \bl{Project}{\iz{
    \item has a name
    \item has one or many sources
    \item specify additional properties, and dependencies (references to other projects)
   }}
}

\fr{A class \Cil{Person}}{
\codeview{1}{5}{34}{\tiny}{\codepath{Person/Program.cs}}
}

\frs{10}{Constructors chaining}{
\codeview{1}{5}{37}{\tiny}{\codepath{PersonChaining/Program.cs}}
}

\frs{10}{Playing with libraries (namespace \Cil{System})}{
\codeviewall{\tiny}{\codepath{PlayWithLibraries/Program.cs}}
}

\frs{10}{State, Getters and Setters}{
    \bl{An object state}{\iz{
        \item an object carries a state, in the form of a structure set of data
        \item internally this is represented by a set of named and typed fields, which are private
        \item externally this is represented by a set of named and typed ``properties''
        \item such properties may or may not overlap with fields
        \item to make such properties accessible to clients, specific methods are needed
    }}
    \bl{Getters and Setters}{\iz{
        \item a common solution in OOP (will see C\# will improve it)
        \item a getter is method \cil{GetXYZ} with 0-args, returning the property \cil{XYZ}'s value, and typically causing no side-effect
        \item a getter is a method \cil{SetXYZ} taking the property \cil{XYZ}'s value and returning nothing
        \item properties that one only wants to read have no setter, and vice-versa for getters
    }}
}

\frs{5}{\Cil{Person} with Getters and Setters}{
\codeview{1}{18}{48}{\tiny}{\codepath{PersonGetters/Program.cs}}
}

\fr{Client code for \Cil{Person}}{
\codeview{2}{7}{16}{\scriptsize}{\codepath{PersonGetters/Program.cs}}
}

\fr{Expression-bodied members}{
    \bl{Syntax: \texttt{<member> => expression; }}{\iz{
        \item can be used for methods and constructors
        \item when their body is a single return of an expression, or just a single statement...
        \item you can directly indicate the signature, \texttt{=>}, and that expression/statement
        \item it makes your programs more short and readable: use them!
    }}
}

\fr{\Cil{Person} with Expression-bodied methods}{
\codeview{1}{18}{39}{\ssmall}{\codepath{PersonExpBody/Program.cs}}
}

\fr{Immutability}{
    \bl{Design for immutability}{\iz{
        \item by choosing which property has a Setter we can decide that there is information that cannot be changed, and this is important to avoid clients to badly affect the behaviour of our objects
    }}
    \bl{Readonly fields}{\iz{
        \item the same has to be done for fields: if a field is initialised at construction time and then never changed, we shall use modifier \cil{readonly}
        \item this enhance clarity of programs, and the compiler check we do not alter such fields
    }}
}


\frs{15}{Properties}{
    \bl{Improving over Get/Set accessors}{\iz{
        \item C\# introduces a programming construct for properties
        \item a property is directly perceived by the client as a sort of field (starting with uppercase)
        \item internally to a class, a property is actually a getter and/or setter with special syntax
    }}
    \bl{Notation}{\iz{
        \item \cil{public <type> <name>\{ get \{...\} set \{...\} \}}
        \item the body of \cil{get} should return a value
        \item the body of \cil{set} can use a special variable \cil{value}
        \item for both we can use expression-bodied get/set
        \item can use expression-bodied readonly property in one line
    }}
    \bl{The special case of auto-implemented properties}{\iz{
        \item if the body of \cil{get} and \cil{set} are entirely skipped, a field with same name of the property is implicitly defined
    }}
}

\fr{\Cil{Person} with Properties}{
\codeview{1}{5}{33}{\tiny}{\codepath{PersonProperties/Program.cs}}
}

\frs{5}{\Cil{Person} with Properties: playing with properties}{
\codeview{1}{20}{50}{\tiny}{\codepath{PersonPropertiesPlay/Program.cs}}
}

\fr{Playing with properties: client code}{
\codeview{2}{7}{18}{\scriptsize}{\codepath{PersonPropertiesPlay/Program.cs}}
}

\section{Some specific C\#/.NET mechanisms}
\renewcommand{\codepath}[1]{../../code/lecture-02/#1}


\end{document}
