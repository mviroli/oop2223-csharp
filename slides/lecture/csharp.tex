\documentclass[presentation]{beamer}
\usepackage{common}
\usepackage{arydshln}

\newcommand{\cscat}[1]{$\langle\text{{\itshape#1}}\rangle$}
\newcommand{\csopt}[1]{{\itshape[#1]}}
\newcommand{\csalt}[1]{{\itshape(#1)}}
\newcommand{\op}[1]{\alert{`\texttt{#1}'}}
\newcommand{\operand}[1][\ldots]{{\normalcolor#1}}
\newcommand{\literal}[1]{\texttt{\alert{#1}}}
\newcommand{\bs}{$\backslash$}

\title[\lecturecode{C\#}]{Object-oriented programming in C\#}


\begin{document}

\AtBeginSubsection[]
{
  \begin{frame}<beamer>
    \frametitle{Next In Line\ldots}
    \tableofcontents[currentsection,currentsubsection]
  \end{frame}
}

\frame[label=coverpage]{\titlepage}

\fr{Outline}{
  \bl{Parts}{\en{
    \item Basic OO concepts in C\#
    \item Some specific C\#/.NET mechanisms
    \item Encapsulation, interfaces
    \item Inheritance
    \item Generics
    \item Exceptions and some key C\# libraries
    \item Functional programming in C\#
    \item Other programming mechanisms
  }}
}

\section{Basic OO in C\#}
\newcommand{\codepath}[1]{../../code/lecture-01/#1}

\frs{5}{Brief introduction to C\#}{
  \bl{C\# and .NET}{\iz{
    \item Designed by Anders Hejlsberg around 2000 at Microsoft
    \item Is part of the .NET initiative, designed to compile over the Common Language Infrastructure (CLI)
    \item Current version is 9.0, released in 2020 for .NET 5.0
    \item Mono is a free, open-source compiler and runtime environment
    \item Initially developed as very similar to Java, then somewhat diverged
    \item Essentially, C\# took a different path than Java in following Scala
    \item Shall in these slides refer to ``mainstream/standard OOP'' to mean the intersection of Java/C\#
    }}
}

\frs{5}{.NET}{
  \bl{Main elements}{\iz{
    \item .NET started as a ployglot framework since its beginning
    \item C\# is by far the mostly used language
    \item Concepts replicate Java and JVM: CIL/bytecode, CLR/JVM, and so on
    \item As a key difference, .NET initially targets Microsoft Windows
    }}
  \fg{height = 0.4\textheight}{img/dotnet-overview.pdf}
}


\begin{frame}{\dotnet Platform -- Present vs. Past}
 \bl{Past to Present}{\iz{
        \item Before \dotnet 5 there used to be three major implementations of the \emph{class library}:
        %
        \begin{description}
            \item[\dotnet Framework] | Windows-specific, full-featured, targetting desktop and web applications
            \item[\dotnet Core] | multi-platform (Win, Mac, Linux), less-featured, targetting desktop and web applications
            \item[Xamarin] | mobile-oriented (Android, iOS, Mac OS) 
        \end{description}

        \vfill

        \item Since \dotnet 5, implementations are aligned
    }}
    \bl{In these slides}{
        Stick to \alert{\dotnet Core 3.1}, to maximise interoperability and to avoid compatibility issues
    }
\end{frame}


\frs{20}{Features of C\#}{
  \bl{Ingredients}{\iz{
    \item C-like language: the imperative and structured part are very similar
    \item Java-like language: essentially very similar to Java, specially at the beginning
    \item Static and strong typing: types are checked at run-time, preventing ill-typed operations
    \item Object-orientation: object by references, automatic garbage collection
    \item Functional-orientation: extension methods, generics, delegates, lambdas
    }}
  \bl{Philosophy}{\iz{
    \item aiming at high expressiveness and richness, though become a rather ``large'' language
  }}
  \fg{height = 0.4\textheight}{img/sizes.png}
}


\fr{C\# types: we start with Simple Types and Class Types}{
  \fg{height = 0.6\textheight}{img/type-system.pdf}
}
  
\frs{20}{Variables: initialisation and inference}{
  \bl{On variables -- essentially as in Java}{\iz{
  \item Same rules on scoping, and assignment
  \item Similar distinction between primitive and class types
  \item Similar naming conventions for variables
  \item \texttt{null} is assignable to variables of reference types
  \item Can use \texttt{var} to declare a variable with type to be inferred
  \item Keywords (\cil{int}, \cil{bool}, \cil{string}, \cil{object}) map to Library Value Types or Classes
  }}
  \codeview{3}{9}{24}{\scriptsize}{\codepath{InitialExamples/Initialisation.cs}}
}

\begin{frame}{\dotnet Built-in Types}\centering
    \begin{table}[]
        \resizebox{\textwidth}{!}{
        \begin{tabular}{c|c|c|c|c}
            \textbf{Name}    & \textbf{Keyword} & \textbf{Category} & \textbf{Size} & \textbf{Description}  \\
            \hline\hline
            \texttt{Boolean} & \texttt{bool}    & \emph{val} & 1             & either \texttt{true} or \texttt{false} \\
            \texttt{Char}    & \texttt{char}    & \emph{val} & 2             & UTF-16 characters \texttt{`U+0000'} \ldots \texttt{`U+FFFF'} \\
            \texttt{Byte}    & \texttt{byte}    & \emph{val} & 1             & integers in $0 \ldots (2^{8}-1)$ \\
            \texttt{SByte}   & \texttt{sbyte}   & \emph{val} & 1             & integers in $-2^{7} \ldots (2^{7}-1)$ \\
            \texttt{Int16}   & \texttt{short}   & \emph{val} & 2             & integers in $-2^{15} \ldots (2^{15}-1)$ \\
            \texttt{UInt16}  & \texttt{ushort}  & \emph{val} & 2             & integers in $0 \ldots (2^{16}-1)$ \\
            \texttt{Int32}   & \texttt{int}     & \emph{val} & 4             & integers in $-2^{31} \ldots (2^{31}-1)$ \\
            \texttt{UInt32}  & \texttt{uint}    & \emph{val} & 4             & integers in $0 \ldots (2^{32}-1)$ \\
            \texttt{Int64}   & \texttt{long}    & \emph{val} & 8             & integers in $-2^{63} \ldots (2^{63}-1)$ \\
            \texttt{UInt64}  & \texttt{ulong}   & \emph{val} & 8             & integers in $0 \ldots (2^{64}-1)$ \\
            \texttt{Float}   & \texttt{float}   & \emph{val} & 4             & abs in $1.5\times 10^{-45} \ldots 3.4\times 10^{38}$ \\
            \texttt{Double}  & \texttt{double}  & \emph{val} & 8             & abs in $5.0\times 10^{-324} \ldots 1.7\times 10^{308}$ \\
            \texttt{Decimal} & \texttt{decimal} & \emph{val} & 16            & abs in $1.0\times 10^{-28} \ldots 7.9228 \times 10^{28}$ \\
            \texttt{Object}  & \texttt{object}  & \emph{ref} & O(1)          & anything \\
            \texttt{String}  & \texttt{string}  & \emph{ref} & O($n$)   & sequences of $n$ UTF-16 characters \\
        \end{tabular}
        }
    \end{table}

    \tiny
    (cf. \url{https://docs.microsoft.com/dotnet/csharp/language-reference/builtin-types/built-in-types})
\end{frame}


\frs{5}{C\# classes}{
  \bl{The core of OOP is essentially as in Java} {\iz{
    \item Classes, methods, fields, and constructors have same syntax and semantics
    \item Class instantiation, method invocation, field access have same syntax and semantics
    \item Static, non-static fields/methods have same syntax/semantics
    \item Structured programming constructs (if/while) have same syntax/semantics
    \item A source file must define the namespace (similar to Java package)
    \item Syntax for calling a constructor from another constructor is different
  }}
  \bl{Formatting}{\iz{
    \item Slightly different conventions on formatting braces 
    \item Methods start with an uppercase, fields with an underscore
    \item \myurl {https://docs.microsoft.com/en-us/dotnet/csharp/programming-guide/inside-a-program/coding-conventions}
  }}
}

\frs{5}{\Cil{Point3D}}{
\codeview{0}{2}{34}{\tiny}{\codepath{Point3D/Program.cs}}
}

\fr {C\# executable programs} {
   \bl {Building blocks of C\# software} {\iz {
     \item class libraries shipped with .NET
     \item possibly other external libraries 
     \item a set of classes that make up the application we build (like \cil{Point3D)}
     \item at least one of these classes has a special method \cil{Main}
     \item a \cil {Main} is the entry point of a program
   }}
   \bl{The \cil {Main} must have the following declaration: }{\iz {
   \item \cil {public static void Main() \{.. \}}
    \item there could be variant with different inputs, outputs, and visibility, but we won't see them now
    \item it is key it is call \cil{Main} and is \cil{static}
    \item \cil{static} means this method is ``shared'' among all objects, and is conceptually called to the class, not to the object
   }}
}

\frs{15}{Typical structure of an executable project} {
   \bl{Entry point class} {\iz {
     \item it contains the \cil{Main} method
     \item typically, it contains only that method
   }}
   \bl{Other classes}{\iz{
     \item contain the various application classes
   }}
   \bl{Source files}{\iz{
    \item have \cil{.cs} extension
    \item start with ``\cil{using}'' clauses to declare other classes they use
    \item declared one or more classes, enclosing them in \cil{namespace}s
    \item a \cil{namespace} is a ``module'' giving a context to the class
   }}
   \bl{Project}{\iz{
    \item has a name
    \item has one or many sources
    \item specify additional properties, and dependencies (references to other projects)
   }}
}

\fr{A class \Cil{Person}}{
\codeview{1}{5}{34}{\tiny}{\codepath{Person/Program.cs}}
}

\frs{10}{Constructors chaining}{
\codeview{1}{5}{37}{\tiny}{\codepath{PersonChaining/Program.cs}}
}

\frs{10}{Playing with libraries (namespace \Cil{System})}{
\codeviewall{\tiny}{\codepath{PlayWithLibraries/Program.cs}}
}

\frs{10}{State, Getters and Setters}{
    \bl{An object state}{\iz{
        \item an object carries a state, in the form of a structure set of data
        \item internally this is represented by a set of named and typed fields, which are private
        \item externally this is represented by a set of named and typed ``properties''
        \item such properties may or may not overlap with fields
        \item to make such properties accessible to clients, specific methods are needed
    }}
    \bl{Getters and Setters}{\iz{
        \item a common solution in OOP (will see C\# will improve it)
        \item a getter is method \cil{GetXYZ} with 0-args, returning the property \cil{XYZ}'s value, and typically causing no side-effect
        \item a getter is a method \cil{SetXYZ} taking the property \cil{XYZ}'s value and returning nothing
        \item properties that one only wants to read have no setter, and vice-versa for getters
    }}
}

\frs{5}{\Cil{Person} with Getters and Setters}{
\codeview{1}{18}{48}{\tiny}{\codepath{PersonGetters/Program.cs}}
}

\fr{Client code for \Cil{Person}}{
\codeview{2}{7}{16}{\scriptsize}{\codepath{PersonGetters/Program.cs}}
}

\fr{Expression-bodied members}{
    \bl{Syntax: \texttt{<member> => expression; }}{\iz{
        \item can be used for methods and constructors
        \item when their body is a single return of an expression, or just a single statement...
        \item you can directly indicate the signature, \texttt{=>}, and that expression/statement
        \item it makes your programs more short and readable: use them!
    }}
}

\fr{\Cil{Person} with Expression-bodied methods}{
\codeview{1}{18}{39}{\ssmall}{\codepath{PersonExpBody/Program.cs}}
}

\fr{Immutability}{
    \bl{Design for immutability}{\iz{
        \item by choosing which property has a Setter we can decide that there is information that cannot be changed, and this is important to avoid clients to badly affect the behaviour of our objects
    }}
    \bl{Readonly fields}{\iz{
        \item the same has to be done for fields: if a field is initialised at construction time and then never changed, we shall use modifier \cil{readonly}
        \item this enhance clarity of programs, and the compiler check we do not alter such fields
    }}
}


\frs{15}{Properties}{
    \bl{Improving over Get/Set accessors}{\iz{
        \item C\# introduces a programming construct for properties
        \item a property is directly perceived by the client as a sort of field (starting with uppercase)
        \item internally to a class, a property is actually a getter and/or setter with special syntax
    }}
    \bl{Notation}{\iz{
        \item \cil{public <type> <name>\{ get \{...\} set \{...\} \}}
        \item the body of \cil{get} should return a value
        \item the body of \cil{set} can use a special variable \cil{value}
        \item for both we can use expression-bodied get/set
        \item can use expression-bodied readonly property in one line
    }}
    \bl{The special case of auto-implemented properties}{\iz{
        \item if the body of \cil{get} and \cil{set} are entirely skipped, a field with same name of the property is implicitly defined
    }}
}

\fr{\Cil{Person} with Properties}{
\codeview{1}{5}{33}{\tiny}{\codepath{PersonProperties/Program.cs}}
}

\frs{5}{\Cil{Person} with Properties: playing with properties}{
\codeview{1}{20}{50}{\tiny}{\codepath{PersonPropertiesPlay/Program.cs}}
}

\fr{Playing with properties: client code}{
\codeview{2}{7}{18}{\scriptsize}{\codepath{PersonPropertiesPlay/Program.cs}}
}

\section{Some specific C\# mechanisms}
\renewcommand{\codepath}[1]{../../code/lecture-02/#1}

\subsubsection{Arrays}

\begin{frame}{\dotnet Arrays}
    \begin{block}{Array types}
        \begin{description}
            \item[\texttt{\textit{T}\alert{[]}}] denotes the \alert{array of \texttt{\textit{T}}} type
            \item[\texttt{\textit{T}\alert{[][]}}] denotes the \alert{array of \emph{arrays of} \texttt{\textit{T}}} type
            \item[\texttt{\textit{T}\alert{[][][]}}] denotes the \alert{array of \emph{arrays of arrays of} \texttt{\textit{T}}} type
            \item[\texttt{\textit{T}\alert{[][,]}}] denotes the \alert{array of \emph{2-dimensional arrays of} \texttt{\textit{T}}} type
            \item[\texttt{\textit{T}\alert{[,,][]}}] denotes the \alert{3-dimensional array of \emph{arrays of} \texttt{\textit{T}}} type
        \end{description}
    \end{block}
\end{frame}
\begin{frame}[shrink=5]{\dotnet Arrays}
    \begin{block}{Arrays features}
        \begin{itemize}
            \item All array types are \alert{reference} types
            %
            \begin{itemize}
                \item arrays of value types are reference types as well
            \end{itemize}

            \item All array types are subtypes of the \texttt{Array} class

            \item Arrays are constructed by sizes, i.e. $D_1, \ldots, D_N$ are user-provided
            %
            \begin{itemize}
                \item so memory can be contigously allocated
                \item items are initialised to their default values
            \end{itemize}

            \item All array types come with 3 useful properties/methods:
            %
            \begin{description}
                \item[\texttt{Rank}] returning the total amount of dimensions of the array (i.e. $N$)
                \item[\texttt{Length}] returning the total amount of items in the array (i.e. $D_1 \times \ldots \times D_N$)
                \item[\texttt{GetLength($i$)}] returning the total amount of items along the $i$-th dimension (i.e. $D_i$)
            \end{description}

            \item Access to items is performed via the indexed-access operator:
            %
            \begin{center}
                \op{\operand[array][\operand[index$_1$, \ldots, index$_N$]]}  
            \end{center}
        \end{itemize}
    \end{block}
\end{frame} 

\begin{frame}[allowframebreaks]{Array Types Instantiation}
    \begin{block}{Constructors for $N$-dimensional Arrays of \texttt{\textit{T}}}
        \begin{center}\ttfamily
            \textit{T}[\alert{,,}\ldots{}] \cscat{Var Name} = \alert{new} \textit{T}[\alert{$D_1$}, \alert{$D_2$}, \ldots];
        \end{center}
        %
        \begin{itemize}
            \item Number of commas in the left-hand side: $N-1$
            \item Number of sizes in the right-hand side: $N$
        \end{itemize}
    \end{block}
    \begin{block}{Literal Array Expressions for $N$-dimensional Arrays of \texttt{\textit{T}}}
        \begin{center}\ttfamily
            \textit{T}[\alert{,,}\ldots{}] \cscat{Var Name} = \alert{new} \textit{T}[\alert{,,}\ldots{}] 
                \alert{\{\ldots\{} \cscat{Item$_1$}, \cscat{Item$_2$}, \ldots  \alert{\}\ldots\}};
        \end{center}
        %
        \begin{itemize}
            \item Number of commas in the left-hand side: $N-1$
            \item Number of nesting levels of braces in the right-hand side: $N$
            \item Repeating \texttt{\textit{T}[\alert{,,}\ldots{}]} may be avoided in the right-hand side
        \end{itemize}
    \end{block}

    \framebreak
    
    \codeview{3}{9}{24}{\tiny}{\codepath{Snippets/Snippet6Arrays.cs}}

\end{frame}

\begin{frame}[allowframebreaks]{Accessing Arrays Items}
    \codeview{2}{27}{41}{\tiny}{\codepath{Snippets/Snippet6Arrays.cs}}

    \codeview{2}{43}{52}{\tiny}{\codepath{Snippets/Snippet6Arrays.cs}}

\end{frame}

\subsubsection{Nullables}

\begin{frame}[shrink=5]{\dotnet Nullables}
    \bl{Nullable types definition}{\iz{
        \item let \texttt{\textit{T}} by a \alert{value} type of any sort, then \texttt{\textit{T}\alert{?}} denotes the \alert{nullable \texttt{\textit{T}}} type

        \item A nullable type \texttt{\textit{T}?} can be defined as \texttt{\textit{T}} $\cup$ \{ \texttt{null} \}.
        %
        \item A variable of type \texttt{\textit{T}?} can be assigned with any admissible value of \texttt{\textit{T}}, \alert{or} with \texttt{null}
        }
        
        {\tiny (cf. \url{https://docs.microsoft.com/dotnet/csharp/language-reference/builtin-types/nullable-value-types})}
    }
    \begin{block}{Nullables features}
        \begin{itemize}
            \item All nullable types are \alert{value} types

            \item The notation \texttt{\textit{T}?} is another way of writing \texttt{Nullable<\textit{T}>}
            %
            \item All nullable types come with some useful properties:
            %
            \begin{description}
                \item[\texttt{HasValue}] returning null if the object is null
                \item[\texttt{Value}] returning the non-null value, if present
            \end{description}

            \item When non-null, nullable-type variables behave like they non-nullable counterparts
        \end{itemize}
    \end{block}
    
\end{frame}

\begin{frame}{Nullable Types Operators}
    \begin{block}{}
        \begin{itemize}
            \item Operator \cil{??} gets the value or a default if null

            \item Operator \cil{?.} calls a method on a nullable only if not null, otherwise it does nothing and yields null
        \end{itemize}
    \end{block}

    \codeview{3}{9}{19}{\scriptsize}{\codepath{Snippets/Snippet7Nullables.cs}}
\end{frame}

\section{Encapsulation, interfaces}
\renewcommand{\codepath}[1]{../../code/lecture-03/#1}

\subsection{Encapsulation, and properties}

\frs{5}{Encapsulation}{
  \bl{Two crucial ingredients of OO programming}{\en{
    \item Packing data + functions to manipulate it
    \item Information hiding via careful access control
  }}
  \bl{Philosophy}{\iz{
    \item Each class declares \cil{public} only those (few) methods/properties/constructors necessary to interact with (or create) its instances
    \item The rest (which therefore includes mere implementation aspects) is \cil{private}{\iz{
      \item methods/constructors/properties for internal use only
      \item{\bf{all}} fields (i.e. internal status)
    }}
  }}
  \bl{Encapsulation and dependencies}{In this way the ``client'' is weakly influenced by possible future modifications concerning mere implementation aspects.
  }
}

\frs{5}{A basic case: class \Cil{Counter}}{
    \codeview{1}{18}{40}{\scriptsize}{\codepath{Counter/Program.cs}}
}

\fr{A basic case: usage of class \Cil{Counter}}{
    \codeview{1}{5}{16}{\scriptsize}{\codepath{Counter/Program.cs}}
}


\fr{Encapsulation is preserved by properties!}{
    \codeview{1}{5}{25}{\ssmall}{\codepath{CounterProperties/Program.cs}}
}

\fr{A transparent modification to the \Cil{Counter} implementation}{
    \codeview{1}{28}{47}{\ssmall}{\codepath{CounterProperties/Program.cs}}
}

\fr{A final bit on properties: object initializers}{
    \codeview{1}{5}{32}{\ssmall}{\codepath{ObjectInitializer/Program.cs}}
}

\fr{Properties vs methods vs fields: recap}{
    \bl{``Properties are just methods''}{\iz{
        \item a read-only property is essentially a Getter
        \item a read-write property is essentially a pair of Getter and Setter method
        \item most discussions in the following focus on methods, and applicability to properties naturally derive
    }}
    \bl{``Properties define a nice abstraction''}{\iz{
        \item from the design viewpoint, public properties are much nicer than Getters/Setters, which are still the OOP standard
    }}
    \bl{``Properties can replace fields''}{\iz{
        \item as an implementation mechanism, auto-implemented properties are a good replacement for fields
        \item but this is just matter of internal implementation
    }}
}

\subsection{Interfaces}


\frs{10}{C\# \Cil{interfaces}}{
  \bl{What is an \cil{interface}}{\iz{
    \item It is a new declarable \alert{reference type} (like classes)
    \item It has a name, and includes a set of method signatures (and properties)
    \item It cannot be used to create objects the \cil{new} operator
  }}
  \bl{An \cil{interface} \cil{I} can be ``implemented'' by a class}{\iz{
    \item Through a class \cil{C} that explicitly declares it (\cil{class C : I \{.. \}})
    \item \cil{C} will define (the body of) all methods declared in \cil{I}
    \item An instance object of \cil{C}, will have the usual \cil{C} type, but also \cil{I}
    \item namely, type \cil{C} is a subtype of \cil{I}
    \item C\# convention for interface names: \cil{IDevice}, \cil{IPerson}, \dots
    \item later versions of C\# provide default methods for interfaces
  }}
  \bl{Substitutability}{\iz{
    \item As usual in OOP: an object created by a class implementing an interface can be passed to where an element of the interface is expected
  }}
}

\fr{Interface \Cil{IDevice}}{
\bx{\cil{IDevice} introduce a contract for devices: they provide services to be switched on, switched off, and to check if they are on.}
\codeview{1}{5}{12}{\small}{\codepath{SmartHome/Program.cs}}
}

\frs{5}{Two lamp implementations of \Cil{Device}}{
  \codeview{1}{14}{38}{\ssmall}{\codepath{SmartHome/Program.cs}}
}

\frs{5}{Multiple implementation}{
  \bl{Multiple implementation}{Possible declaration: \cil{class C : I1, I2, I3 \{.. \}}
  \iz{
    \item A class \cil{C}  implements \cil{I1} and \cil{I2} and \cil{I3}
    \item The class \cil{C}  ust provide a body for all methods of \cil{I1}, all those of \cil{I2}, all those of \cil{I3}{\iz{
      \item if \cil{I1}, \cil{I2}, \cil{I3} had common methods there would be no problem, each one should be implemented only once
    }}
    \item Instances of \cil{C} have type \cil{C}, but also types \cil{I1}, \cil{I2} and \cil{I3}
  }}
  \bl{Extension}{Possible declaration: \cil{interface I : I1, I2, I3 \{.. \}}
  \iz{
    \item An interface \cil{I} defines certain methods, in addition to those of \cil{I1}, \cil{I2}, \cil{I3}
    \item A class \cil{C} that implements \cil{I} must provide a body for all methods indicated in \cil{I}, plus all those of \cil{I1}, all those of \cil{I2}, and all those of \cil{I3}
    \item Instances of \cil{C} have type \cil{C}, but also types \cil{I}, \cil{I1}, \cil{I2} and \cil{I3}
  }}
}

\section{Inheritance}
\renewcommand{\codepath}[1]{../../code/lecture-04/#1}

\subsection{Class extension}

\fr{Inheritance}{
  \bx{
    It is a mechanism that allows you to define a new class \alert{specialising} an existing one, that is, ``inheriting'' its members (the private ones are not directly visible), possibly modifying / adding new members, and therefore reusing code already written and tested.
  }
  \bl{Inheritance is a key concept of OOP}{\iz{
    \item It is related to the interface mechanism
    \item It is one of the key elements along with encapsulation and interfaces
    \item It not only affects code reuse, but also the resulting polymorphism
  }}
  \bl{Abstract classes}{\iz{
    \item dealt with in a completely standard way, we won't consider them further here
  }}
}


\fr{Basic example: \Cil{Counter}}{
\codeview{1}{5}{25}{\scriptsize}{\codepath{Counter/Program.cs}}
}

\frs{15}{The need to extend and modify}{
  \bl{The inheritance mechanism can be used for a multicounter}{\iz{
    \item Definition: \cil{class C : D \{.. \}}
    \item The new \cil{C} class inherits all members of \cil{D}{\iz{
      \item The private members are not directly accessible from within \cil{C}
      \item The constructors of \cil{D} must always be rewritten, and should properly call \cil{C}'s
      \item The constructor of a subclass should have the \cil{base} statement, which calls a (non-private) constructor of the parent class
    }}
  }}
  \codeview{1}{18}{29}{\scriptsize}{\codepath{MultiCounterInheritance/Program.cs}}
}


\frs{8}{\Cil{protected} access level }{
  \bl{Usable for the members of a class}{\iz{
    \item It is an intermediate level between \cil{public} and \cil{private}
    \item Indicates that the member (field, method, constructor, property) is accessible from the current class, from a subclass, and from subclasses of subclasses (recursively)
  }}
  \bl{What is it for?}{\iz{
    \item It allows subclasses to access supra-class information that you don't want clients to see
    \item Most often used in retrospect replacing a \cil{private}
    \item Using \cil{protected} fields is to be avoided -- it somewhat breaks encapsulation; should better use \cil{protected} properties/methods
  }}
  \bl{Example class \cil{BiCounter} - bidirectional counter}{\iz{
    \item A counter with also the \cil{Decrement} method
    \item Impossible without making the \cil{Value} accessible also for modification
  }}
}

\fr{\Cil{ExtendibleCounter} and \Cil{BiCounter}}{
  \codeview{1}{5}{34}{\tiny}{\codepath{BiCounter/Program.cs}}
}

\fr{Analogous solution with fields}{
  \codeview{1}{18}{38}{\ssmall}{\codepath{BiCounterFields/Program.cs}}
}

\frs{20}{Overriding}{
  \bl{Extension and modification}{\iz{
    \item When creating a new class by extension, it is very often not enough to add new functionality
    \item Sometimes it is also necessary to modify some of those available, possibly even distorting a bit their original functioning
    \item This can be done by rewriting in the subclass one (or more) of the methods/properties of the superclass -- called an \alert{override}
    \item To do so, methods/properties in the base class must be declared \cil{virtual}, and those in the subclass \cil{override}
    \item If necessary, the rewritten method can invoke the version of the parent using the special receiver \cil{base}
    \item It is possible to ``hide'' a method in the superclass that is not \cil{virtual}, by the modifier \cil{new} -- but this mechanis is optional, and arguably with limited use
    \item A class can be declared \cil{sealed} to prevent extension
    
  }}
  \bl{Example \cil{LimitCounter}}{\iz{
    \item Create a (sealed) counter which, having reached a certain limit, no longer continues
    \item It is necessary to override the \cil{Increment()} method
    \item An additional getter method inspects when the limit is reached
  }}
}

\fr{Using the \Cil{LimitCounter} class}{
  \codeview{1}{5}{18}{\scriptsize}{\codepath{LimitCounter/Program.cs}}
}


\fr{Class \Cil{LimitCounter}}{
  \codeview{1}{20}{41}{\scriptsize}{\codepath{LimitCounter/Program.cs}}
}

\fr{A summary of access modifiers (for types and members)}{
    \bx{Recall that \cil{internal} means ``visible only in this assembly''}
    \bl{Who can access?}{\iz{
        \item \cil{public}: any other code in the same assembly or another assembly that references it
        \item \cil{private}: only by code in the same class
        \item \cil{protected}: only by code in the same class, or in a class that is derived from that class
        \item \cil{internal}: by any code in the same assembly, but not from another assembly
        \item \cil{protected internal}: by any code in the assembly in which it's declared, or from within a derived class in another assembly
        \item \cil{private protected}: only within its declaring assembly, by code in the same class or in a type that is derived from that class
    }}
}

\fr{Class \Cil{Object}}{
    \bl{Implicit extension of \cil{Object}}{\iz{
        \item when a class extends nothing, it is like extending \cil{System.Object}
        \item transitively, this means all classes inherit from \cil{Object}
        \item it provides low-level services for all objects 
        \item we will in the following explain some of them
    }}
    \bl{Method \cil{String ToString()}}{\iz{
        \item it can be overriden to provide a canonical string representation of an object
        \item \cil{Console.Write} uses it if you try to write an object
    }}
}

\subsection{Runtime types}

\fr{Polymorphism with \Cil{object}}{
  \codeview{1}{5}{29}{\ssmall}{\codepath{PolyObject/Program.cs}}
}


\fr{\Cil{UsePerson} and \Cil{Person}}{
  \codeview{1}{5}{34}{\tiny}{\codepath{Person/Program.cs}}
}

\fr{Specialisations of \Cil{Person}}{
  \codeview{1}{36}{58}{\tiny}{\codepath{Person/Program.cs}}
}

\frs{5}{Static type and run-time type}{
  \bl{A duality introduced by subtyping (inclusive polymorphism)}{\iz{
    \item Static type: the data type of an expression that can be inferred by the compiler
    \item Run-time type: the data type of the value (/ object) actually present (could be a subtype of the static one, and can be inspected with \cil{GetType()}){\iz{\item in this case virtual method calls rely on late-binding}}
  }}
  \bl{Example in \cil{PrintAll()} code, inside the \cil{foreach}}{\iz{
    \item Static type of \cil{obj} is \cil{Object}
    \item Run-time type of \cil{obj} varies from time to time: \cil{String}, \cil{Int32}, \dots
  }}
  \bl{Type inspection at run-time}{\iz{
    \item In some cases it is necessary to inspect the type at run-time
    \item The case of the \cil{is} and \cil{as} operators
    \item However, using them is bad practice: it means you have poorly used polymorphism
  }}
}

\fr{Type check and conversion}{
  \codeview{1}{6}{32}{\tiny}{\codepath{PersonRuntime/Program.cs}}
}


\end{document}
